\documentclass{article}
\usepackage{graphics,eurosym,latexsym}
\usepackage{times}
\usepackage{listings}
\lstset{numbers=left,numberstyle=\tiny,stepnumber=5,breaklines=true}
\bibliographystyle{plain}
\oddsidemargin=0cm
\evensidemargin=0cm
\newcommand{\be}{\begin{enumerate}}
\newcommand{\ee}{\end{enumerate}}
\newcommand{\bi}{\begin{itemize}}
\newcommand{\ei}{\end{itemize}}
\newcommand{\I}{\item}
\newcommand{\ty}{\texttt}
\newcommand{\kr}{K_{\rm r}}
\newcommand{\version}{2.0.2}
\textwidth=16cm
\textheight=23cm
\begin{document}

\title{\ty{kr} version \version: Efficient Estimation of Pairwise
  Distances between Genomes}
\author{Mirjana Domazet-Lo\u{s}o$^{1,2}$ and Bernhard Haubold$^1$\\
\small
$^{1}$Max-Planck-Institute for Evolutionary Biology
  Department of Evolutionary Genetics, 24306 Pl\"on, Germany\\
\small
$^{2}$University of Zagreb, Faculty of Electrical Engineering and
  Computing, Zagreb, Croatia}                                   
\maketitle

\section{Introduction}
\ty{kr} is a program for efficiently computing the statistic $\kr$, an
estimator of the pairwise number of
substitutions between long DNA sequences. For details on this statistic see~\cite{hau09:est},
which also accompanies version 1 of \ty{kr}. The algorithmic
improvements implemented in version 2 are described
in~\cite{dom09:eff}. 

\section{Input and Output Formats}
The input consists of DNA sequences in FASTA format. The alphabet of
these sequences is restricted to the four canonical bases \ty{A},
\ty{C}, \ty{G}, and \ty{T}. The output is a
distance matrix in PHYLIP format.

\section{Getting Started}
\ty{kr} is written in C and is
intended to run on any UNIX system with a C compiler and the GNU
Scientific Library installed~\cite{gal05:gnu}. However, please contact
MDL at \ty{mdomazet@evolbio.mpg.de} if you have problems with the
program.
\bi
\I Unpack the software
\begin{verbatim}
tar -xvzf kr_XXX.tgz
\end{verbatim}
where \ty{XXX} indicates the version.
\I Change into the newly created directory
\begin{verbatim}
cd Kr_XXX/Src/Kr
\end{verbatim}
and list its contents
\begin{verbatim}
ls
\end{verbatim}
\I Users of Mac OSX need to edit the file \ty{Makefile} such that the
assignments to the variables \ty{CFLAGS} and \ty{CFLAGS64} are changed
to
\begin{verbatim}
CFLAGS= -O3 -Wall -Wshadow -pedantic -D_GNU_SOURCE                    \
-D_FILE_OFFSET_BITS=64 -std=c99 -DVER32 -DUNIX -I/opt/local/include/  \
-L/opt/local/lib 
\end{verbatim}
and
\begin{verbatim}
CFLAGS64= -O3 -Wall -Wshadow -pedantic -D_GNU_SOURCE          \
-D_FILE_OFFSET_BITS=64 -std=c99 -DUNIX -I/opt/local/include/  \
-L/opt/local/lib 
\end{verbatim}
\I Generate \ty{kr}
\begin{verbatim}
make
\end{verbatim}
\I List its options
\begin{verbatim}
./kr -h
\end{verbatim}
\I Test the program and record its run time
\begin{verbatim}
time ./kr ../../Data/hiv42.fasta
\end{verbatim}
This carries out the standard distance computation. In the case of HIV
sequences, however, it is safe to assume that the GC content does not
vary greatly between sequences. By telling the program to use a global
GC content rather than the GC content of each pair of sequences, the
computation of $\kr$ can be sped up:
\begin{verbatim}
time ./kr -g ../../Data/hiv42.fasta
\end{verbatim}
The difference in run time becomes particularly pronounced for large
data sets.
\ei


\section{Listings}
\lstset{language=c}
The following listings document central parts of the code for \ty{kr}.
\subsection{The Driver Program: \ty{mainKr.c}}
\lstinputlisting{../Src/Kr/mainKr.c}
\subsection{Traverse Suffix Tree: \ty{lcpSubjectTree.c}}
\lstinputlisting{../Src/Kr/lcpSubjectTree.c}
\subsection{Estimate Divergence: \ty{divergence.c}}
\lstinputlisting{../Src/Kr/divergence.c}

\section{Change Log}
\bi
\I Version 2.0.0 (June 27, 2009)
\bi
\I First running version
\ei
\I Version 2.0.1 (July 28, 2009)
\bi
\I Implemented shortcut for $\kr$ computation (\ty{-g})
\ei
\I Version 2.0.2 (October 5, 2009)
\bi
\I Added more detailed error messages
\ei
\ei


\bibliography{references}

\end{document}

